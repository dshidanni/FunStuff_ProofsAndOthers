\documentclass[11pt,letterpaper]{article}
\usepackage[utf8]{inputenc}
\usepackage[top=0.7in, bottom=0.7in, left=0.7in, right=0.7in]{geometry}
\usepackage{amsmath}
\usepackage{amssymb}
\usepackage{graphicx}
\usepackage{hyperref}
\usepackage{tabu}
\usepackage{longtable}
\usepackage{booktabs}
\usepackage{dsfont}
\usepackage{tabularx}
\usepackage{fancyhdr}
\usepackage{enumitem}
\usepackage{multirow}
\usepackage{multicol}
\usepackage[bottom]{footmisc}
\usepackage{bm}
\usepackage{float}
\usepackage[english]{babel}
\usepackage{color}
\usepackage{verbatim}
\usepackage{mathtools}
\usepackage{esvect}
\usepackage{mathabx}
\usepackage{blkarray}

\title{Vandermonde's Identity: an Algebraic Proof}
\author{Danni Shi}
\date{November 2018}

\begin{document}
\maketitle
\noindent An algebraic approach to prove the Vandermonde's Identity:
$$\sum_{x=0}^z \binom{n_1}{x} \binom{n_2}{z-x} = \binom{n_1 + n_2}{z}$$
\noindent Recall the polynomial expansion
$$(k+1)^{n_1 + n_2} = \sum_{z=0}^{n_1 + n_2} \binom{n_1 + n_2}{z}k^z (*)$$
\noindent Also note that $(k+1)^{n_1 + n_2} = (k+1)^{n_1} \cdot (k+1)^{n_2}$, we can have
\begin{align*}
&(k+1)^{n_1 + n_2} = (k+1)^{n_1} \cdot (k+1)^{n_2} = \left(\sum_{i=0}^{n_1} \binom{n_1}{i} k^i\right) \cdot \left(\sum_{j=0}^{n_2} \binom{n_2}{j} k^j\right) \\
&= \left(\binom{n_1}{0} + \binom{n_1}{1}k + \binom{n_1}{2}k^2 + \cdots + \binom{n_1}{n_1}k^{n_1}\right) \cdot \left(\binom{n_2}{0} + \binom{n_2}{1}k + \binom{n_2}{2}k^2 + \cdots + \binom{n_2}{n_2}k^{n_2}\right)\\
&= \binom{n_1}{0} \binom{n_2}{0} + \left(\binom{n_1}{0}\binom{n_2}{1} + \binom{n_1}{1}\binom{n_2}{0}\right)k + \left(\binom{n_1}{0}\binom{n_2}{2} + \binom{n_1}{1}\binom{n_2}{1} + \binom{n_1}{2}\binom{n_2}{0}\right)k^2 + \cdots
\end{align*}
\noindent Let's go back to (*), so now the coefficient for a $k^z$ can also be expressed as
$$\binom{n_1}{0}\binom{n_2}{z} + \binom{n_1}{1}\binom{n_2}{z-1} + \cdots + \binom{n_1}{z}\binom{n_2}{0} = \sum_{x=0}^z \binom{n_1}{z}\binom{n_2}{z-x}$$
\noindent which implies that 
$$\sum_{x=0}^z \binom{n_1}{x} \binom{n_2}{z-x} = \binom{n_1 + n_2}{z}$$

\end{document}
